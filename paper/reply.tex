\documentstyle[11pt]{article}
\topmargin=-.5cm
\textheight=22cm
\oddsidemargin=.0cm
\textwidth=15.8cm
\baselineskip=18pt
\def\be{\begin{equation}}
\def\ee{\end{equation}}
\def\<{\noindent }

\begin{document}

\title{\bf
Reply to the Referee Report -- AAS03163 }
\author{Zhen Pan \quad zhpan@ucdavis.edu\\
Department of Physics, University of California, Davis;\\
Cong Yu \quad cyu@ynao.ac.cn\\
Yunnan Observatories, Chinese Academy of Sciences, Kunming 650011, China;\\
Lei Huang \quad muduri@shao.ac.cn\\
Shanghai Astronomical Observatory, \\
Chinese Academy of Sciences, Shanghai, 200030, China}

\maketitle


\<*****************************************************************************

\<We thank our referee Prof. I. Contopoulos for
his insightful comments which greatly improved this paper. We have
modified relevant parts in this paper according to these comments
and all significant modifications are written in bold fonts
in the revised manuscript. \\


{\bf I was not impressed by this paper. The authors essentially repeat
the discussion of Nathanail \& Contopoulos 2014 on radiation and
particular boundary conditions at infinity and at the horizon,
and apply their numerical algorithm (with assistance from Dr. Nathanail)
to obtain the solution for the particular case of a uniform field at infinity.\\

I found their discussion confusing (the difference between radiation-or ``constraint"- conditions
and particular boundary conditions-e.g. uniform field- at infinity is not very clear).}\\

\< Reply : In the revised version,  we rephrase the discussion as
``But the radiation conditions and boundary conditions are not independent.
For example, the radiation condition (5) uniquely determines the boundary values at horizon
if $\Omega$ and $I$ are specified, and we will use it as the inner boundary condition in our numerical calculation''
(see Section 3.1 for more details).\\

{\bf I did not like their comment that the field at infinity ``is not monopole".
It should be clear that the only thing the solution at infinity knows about is
what is the shape of the outer boundary, not how field lines are distributed inside that boundary
(this is the job of the GS equation to determine): if the solution is free to fill all space,
then the solution becomes asymptotically monopole (in the sense that $B\rightarrow B_r$);
if it is constrained inside a paraboloidal outer boundary, the solution obtains
an asymptotically paraboloidal shape; if the solution is constrained inside a cylinder,
the solution must become asymptotically vertical. In general, this vertical field DOES NOT NEED TO BE uniform.
Setting it to be uniform is a very strong and very particular BOUNDARY condition, which,
as first stated in Nathanail \& Contopoulos 2014, does uniquely determine the overall solution.}\\

\< Reply : We originally misused monopole as field with potential $A_\phi \propto 1-\cos\theta$
which led to the ``is not monopole" confusion as our referee pointed out.
In the modified version, we rephrase as
 ``{\bf field at infinity deviates from $A_\phi \propto 1-\cos\theta$}''. \\

{\bf I suggest that the authors address my comments on boundary and constraint conditions at infinity,
but I do not need to see the paper again. I recommend publication because
I must give them credit for obtaining the solution for the particular limited case of
a uniform field at infinity and for emphasizing that it contains a current sheet.}\\

\< Reply : We address the comments as above.
In addition, we also correct some of our misunderstanding about Nathanail \& Contopoulos 2014' results,
and point out the debates about the validity of the radiation conditions. We hope the revised
manuscript is more readable and is appropriate for publication now.





\end{document}
